%\chapter{Large Scale Bayesian Inference}

\section{Fokker-Planck Approximation (VIS-FP)}\label{app:fp}
  
  The Fokker--Planck equation is a PDE %Please define if appropriate 
  that describes the temporal evolution of the density of a random variable under a (stochastic) gradient flow \parencite{pavliotis2014stochastic}. For a given SDE %Please define if appropriate
  
$$
dz = \mu(z,t)dt + \sigma(z,t)dB_t,
$$
the corresponding Fokker--Planck equation is
$$
\frac{\partial}{\partial t} q_t(z) = -\frac{\partial}{\partial z}\left[ \mu(z,t)q_t(z)\right] + \frac{\partial^2}{\partial z^2} \left[ \frac{\sigma^2(z,t)}{2} q_t(z) \right].
$$
We are interested in converting the SGLD dynamics to a deterministic gradient flow. %(that is, we want to convert a SDE into an ODE such that both gradient flows have the same Fokker-Planck equation). 

\begin{Proposition}
The SGLD dynamics, given by the SDE
$$
dz = \nabla \log p(z)dt + \sqrt{2}dB_t,
$$
have an equivalent deterministic flow, written as the ODE %Please define if appropriate
$$
dz = (\nabla \log p(z) - \nabla \log q_t (z))dt.
$$
\end{Proposition}
\begin{proof}
Let us write the Fokker--Planck equation for the respective flows. For the Langevin SDE, it is 
$$
\frac{\partial}{\partial t} q_t(z) = - \frac{\partial}{\partial z} \bigg[ \nabla \log p(z) q_t(z) \bigg] + \frac{\partial^2}{\partial z^2} \bigg[ q_t(z) \bigg].
$$
On the other hand, the Fokker--Planck equation for the deterministic gradient flow is given by
$$
\frac{\partial}{\partial t} q_t(z) = - \frac{\partial}{\partial z} \bigg[ \nabla \log p(z) q_t(z)\bigg] + \frac{\partial}{\partial z} \bigg[ \nabla \log q_t(z) q_t(z)\bigg].
$$
The result immediately follows, since $ \frac{\partial}{\partial z} \left[ \nabla \log q_t(z) q_t(z)\right] = \frac{\partial^2}{\partial z^2} \left[ q_t(z) \right]$.
\end{proof}
Given that both flows are equivalent, we restrict our attention to the deterministic flow. Its discretization leads to iterations of the form
\begin{equation}\label{eq:deterministic_flow}
z_{t} = z_{t-1} + \eta (\nabla \log p(z_{t-1}) - \nabla \log q_{t-1} (z_{t-1})).
\end{equation}
In order to tackle the last term, we make the following particle approximation. Using a variational formulation, we have 
\begin{align*}
    - \nabla \log q(z) = \nabla \left( - \frac{\delta}{\delta q} \mathbb{E}_q \left[ \log q\right] \right).
\end{align*}

Then, we smooth the true density $q$ convolving it with a kernel $K$, typically the rbf %Please define if appropriate 
kernel, $K(z, z') = \exp \lbrace - \gamma \| z - z' \|^2 \rbrace$, where $\gamma$ is the bandwidth hyperparameter, leading to
\begin{align*}
    \nabla \left( - \frac{\delta}{\delta q} \mathbb{E}_q \left[ \log q\right] \right) &\approx
     \nabla \left( - \frac{\delta}{\delta q} \mathbb{E}_q \left[ \log (q\ast K )\right] \right) \\
     &= \nabla \log (q \ast K) - \nabla \left( \frac{q}{(q \ast K)} \ast K \right).
\end{align*}

If we consider a mixture of Dirac deltas, $q(z) = \frac{1}{M} \sum_{m=1}^M \delta(z - z_m)$, then the approximation is given 
by
$$
- \nabla \log q(z) \approx - \frac{\sum_k \nabla_{z_m} K(z_m, z_n)}{\sum_n K(z_m, z_n)}
- \sum_l \frac{\nabla_{z_m} K(z_m, z_l)}{\sum_n K(z_n, z_l)},
$$
which can be inserted into Equation (\ref{eq:fppp}). Finally, note that
it is possible to back-propagate through this equation; i.e., the gradients of $K$ can be explicitly computed.

%%%%%%%%%%%%%%%%%%%%%%%%%%%%%%%%%%%%
\section{Experiment Details}\label{sec:detail}


\subsection{State-Space Models}\label{app:ss}

\paragraph{Initial experiments.}\label{app:hmm}
For the HMM, both the observation and transition probabilities are categorical distributions, taking values in the domain $\lbrace 0, 1, 2, 3, 4 \rbrace$.

The equations of the DLM are 
\begin{align*}
    z_{t+1} &\sim \mathcal{N}(0.5z_t + 1.0,\sigma_{tr}) \\
    x_{t} &\sim \mathcal{N}(3.0z_t + 0.5, \sigma_{em}).
\end{align*}
with $z_0  = 0.0$.

%\subsubsection{Prediction Task in a Hmm}

\paragraph{Prediction task in a DLM.}
The DLM model comprises a linear trend component plus a seasonal block with a period of 12. The trend is specified as
\begin{align*}
x_t &= z_{level,t} + \epsilon_t \qquad \epsilon_t \sim \mathcal{N}(0, \sigma_{obs}) \\
z_{level,t} &= z_{level,t-1} + z_{slope,t-1} + \epsilon'_t \qquad \epsilon'_t \sim \mathcal{N}(0, \sigma_{level}) \\
z_{slope,t} &= z_{slope,t-1} + \epsilon''_t \qquad \epsilon''_t \sim \mathcal{N}(0, \sigma_{slope}).
\end{align*}

With respect to the seasonal component,
%the main idea is to \emph{cycle the state}: suppose $\theta_t \in \mathbb{R}^p$, with $p$ being the seasonal period. Then, at each timestep, the model focuses on the first component of the state vector:
% $$( \underuparrow{\alpha_1}, \alpha_2, \ldots, \alpha_{p}) \xrightarrow{\text{next period}} ( \underuparrow{\alpha_2}, \alpha_3, \ldots, \alpha_{p}, \alpha_{1}).$$
%Thus,
we specify it through
\begin{align*}
x_t &= F z_t + v_t \qquad v_t \sim \mathcal{N}(0, \sigma_{obs})\\
z_t &= G z_{t-1} + w_t \qquad w_t \sim \mathcal{N}(0, \sigma_{seas})
\end{align*}
where $F$ is a $12$-dimensional vector
$( 1,0,\ldots, 0,0)$ 
and $G$ is the $12\times 12$ matrix 
\begin{equation*}
G = \begin{bmatrix}
0 & 0 & \ldots & 0 & 1 \\
1 &	0 & & 0 & 0 \\
0 & 1 & & 0 & 0 \\
 & & \ddots & & \\
 0 & 0 & & 1 & 0
\end{bmatrix}.
\end{equation*}
\\

Further details are in \textcite{west1998bayesian}.
% OLDE...
\iffalse
\subsubsection{Model Details} The definition of the HMM and the DLM can be found in Figures \ref{fig:hmm} and \ref{fig:dlm}, respectively.

\begin{figure}[!htp]
\centering
\RecustomVerbatimEnvironment{Verbatim}{BVerbatim}{}
\inputminted[fontsize=\scriptsize]{python}{./hmm.tex}
\caption{Model architecture for the HMM.}
\label{fig:hmm}
\end{figure}

\begin{figure}[!htp]
\centering
\RecustomVerbatimEnvironment{Verbatim}{BVerbatim}{}
\inputminted[fontsize=\scriptsize]{python}{./dlm.tex}
\caption{Model architecture for the DLM.}
\label{fig:dlm}
\end{figure}
\fi

\subsection{VAE}

\paragraph{Model details.}

%The VAE model is implemented with PyTorch \parencite{paszke2017automatic}.
The prior distribution $p(z)$ for the latent variables $z \in \mathbb{R}^{10}$ is a standard factorized Gaussian. The decoder distribution $p_\theta(x|z)$ and the encoder distribution (initial variational approximation) $q_{0,\phi}(z|x)$ are parameterized by two feed-forward neural networks, as 
detailed in Figure \ref{fig:arch_vae}.



\paragraph{Hyperparameter settings.}
The optimizer Adam is used in all experiments, with 
la earning rate of $\lambda=0.001$. We~also set $\eta = 0.001$. We train for 15 epochs (fMNIST) and 20 epochs (MNIST) to achieve a performance 
similar to the VAE in \parencite{pmlr-v89-titsias19a}. For the VIS-5-10 setting, we train only for 10 epochs to allow a fair computational comparison in terms of similar computing times.

\subsection{cVAE}

%\subsubsection{Probabilistic Graphical Model}



\paragraph{Model details.}

%The cVAE model is implemented with PyTorch \parencite{paszke2017automatic}.
The prior distribution $p(z)$ for the latent variables $z \in \mathbb{R}^{10}$ is a standard factorized Gaussian. The decoder distribution $p_\theta(x|y,z)$ and the encoder distribution (initial variational approximation) $q_{0,\phi}(z|x,y)$ are parameterized by two feed-forward neural networks whose details can be found in Figure \ref{fig:arch}.
Equation (\ref{eq:mc_cvae}) is approximated with one MC sample from the variational approximation in all experimental settings, {as it allowed fast inference times while offering better results.}


\begin{figure}[h]
\RecustomVerbatimEnvironment{Verbatim}{BVerbatim}{}
\inputminted[fontsize=\scriptsize]{python}{img/arch_vae.tex}
\caption{Model architecture for the VAE.}
\label{fig:arch_vae}
\end{figure}
\unskip
\begin{figure}[h]

\RecustomVerbatimEnvironment{Verbatim}{BVerbatim}{}
\inputminted[fontsize=\scriptsize]{python}{img/arch.tex}
\caption{Model architecture for the cVAE.}
\label{fig:arch}
\end{figure}


\paragraph{Hyperparameter settings.}
The optimizer Adam was used in all experiments, with a learning rate of $\lambda=0.01$. We set the initial $\eta = 5 \times 10^{-5}$.




\iffalse
\section{Additional Experiments}

We now repeat the experiments but with also tuning \texttt{lr\_inn}.

Note this is not worth it in the DLM/HMM case since it goes from 2 params to 3 params (adds a lot of overhead), whereas in the VAE case the overhead is negligible.



\begin{table}[!ht]
\centering
\caption{Test log-likelihood on binarized MNIST and fMNIST.}\label{tbl:iwhvae}
\begin{tabular}{lcc}
\cline{1-3}
\textbf{Method}   & \textbf{MNIST}                             & \textbf{fMNIST}   \\ \cline{1-3}
\multicolumn{3}{c}{\small Results from \parencite{pmlr-v89-titsias19a}}       \\
    UIVI          & $-94.09$ &  $-110.72$ \\
    SIVI          & $-97.77$ &  $-121.53$ \\
    VAE          & $-98.29$ &  $-126.73$ \\
    \cline{1-3}
    %VIS-10-20 (this paper)     & --- & $\bm{-106.90}$  \\
    VIS-5-10 (this paper)     & $\bm{-65.55 \pm 0.18}$ & $\bm{-89.97 \pm 0.57}$  \\
    %VIS-0-20 (this paper)    & --- & $-121.11$  \\
    VIS-0-10 (this paper)     & $-98.13 \pm 0.13$ & $-121.39 \pm 0.49$  \\
    %VAE + RealNVP     & --- & ---  \\
    %VAE + IAF         & --- & --- \\
    VAE (VIS-0-0)              & $-100.92 \pm 0.09$ & $-125.68 \pm 0.50$ \\
\end{tabular}
\end{table}

Add full table with results from main and also training losses?
\fi

\iffalse
\section{Further Work}

\subsection{Gradient Estimation}

Instead of the Delta approximation, we may use the dual formulation of the KL divergence as done in \url{https://arxiv.org/abs/1908.11527}. We have 
$$
KL(q(z|x) | p(z)) = \max_{\nu} \mathbb{E}_{q(z|x)} \nu_{\psi}(x,z) -  \mathbb{E}_{p(z)} \exp (\nu_{\psi}(x,z)).
$$

With this approach, we need to use an additional neural network (it is like the discriminator used in AVB).


\subsection{Exploring the Bi-Level Formulation}

Recall the manuscript notes where we can use different divergences at each level.
Also, if we use a parameterized divergence at the inner loop (like the $\alpha$-divergence) we could optimize wrt $\alpha$ using the gradient from the outer loss.

\subsection{Structure for Next Paper}

\begin{enumerate}
    \item General framework as a bi-level optimization with two divergences (potentially parameterized): highlight the transition from points to distributions (follow the paper of \texttt{higher})
    \item Concrete examples:
    \begin{enumerate}
        \item VIS for tuning sampler parameters: KL, KL.
        \item Tuning $\alpha$: KL, $\alpha$.
        \item Some with OPVI: try to related with parameterized f-divergence.
        \item Try to think an application: for example better VI for the neural Kalman filter.
    \end{enumerate}
    \item Gradient estimation techniques (maybe this should be particularized for each setting):
    \begin{enumerate}
        \item Delta approximation
        \item Discriminator network
        \item Dual formulation
    \end{enumerate}
\end{enumerate}


$$
\alpha^* = \mbox{argmin}_{\alpha} l_o(\alpha, \beta^*)
$$
st
$$
\beta^* = \mbox{argmin}_{\beta} l_i(\alpha, \beta).
$$

The inner problem could be one of these \url{https://papers.nips.cc/paper/6208-renyi-divergence-variational-inference.pdf}, \url{http://proceedings.mlr.press/v48/hernandez-lobatob16.pdf}, and we can tune the $\alpha$ using the same loss (but maybe over a validation set).

We have to make the notation clearer.
\fi